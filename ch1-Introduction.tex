\chapter{Introduction}
  Will present the results of a search for new physics in dilepton events near the Z mass.
\section{The standard model of particle physics}
  Came into birth in the 60s by the work of higgs, weinberg, sheldon glashow, etc...
  \subsection{The evolution of particle physics}
    Parton model of protons
    Std. model has predicted particle and condescend matter phenomenology. The latest prediction that worked is the Higgs and it's mass couplings/decays
    Fine structure constant, W/Z predictions, 3rd generation
  \subsection{The framework of Quantum Field Theory}
    Lagrangians and Feynman rules, show the S.M. Lagrangian, explain terms. Particles are fields, etc...
  \subsection{Problems with the Standard Model}
    Dark matter, dark energy, neutrino masses, gravitation
    Curiosities: Why 3 generations, should there be naturalness, unstable universe
    Deviations: TTH xsec, muon magnetic moment
  Leptons in this document refer to light leptons, electrons or muons.
\section{Supersymmetry as the Deus Ex Machina}
  \subsection{Why SUSY?}
    GUT miracle, coleman-mandula, naturalness
  \subsection{R-parity} \label{sec:r-parity}
    Proton lifetime, dark matter candidate
  \subsection{Simplified Models}
    What's the idea, MSSM, pMSSM, SMS (using only one new particle).
    GMSB: https://arxiv.org/pdf/hep-ph/9707450.pdf, https://arxiv.org/pdf/hep-ph/9801271.pdf
  \subsection{Parameter space}
    Not all susy needs to be at EWK scale, why do we think it should be? What can we actually rule out with these models?
\section{Why focus on the Z with \MET final state?}
  Should basically copy from Chapter 3.
  \subsection{Past results}
