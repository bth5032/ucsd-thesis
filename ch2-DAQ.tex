\chapter{The acquisition of data}

\section{The LHC}

\begin{figure}[h!]
  \centering
  \includegraphics[width=.7\textwidth]{figures/lhc_decay_modes.jpg}
  \caption{Production Cross Sections for the LHC}
  \label{fig:lhc_decay_modes}
\end{figure}

\section{The CMS Detector}

\section{Physics Objects}
\subsection{Particle flow and event clustering}
\subsection{Vertex Selection}
\subsection{B-Tagging}
\subsection{Electron Measurement Pipeline} \label{sec:electron_measurement_pipeline}
Tracker, ecal, distinguishing between electrons and muons. Cite the CMS paper on electron energy reco..
\subsection{Muon Measurement Pipeline} \label{sec:muon_measurement_pipeline}
Tracker muons, global muons, muon system
\subsection{Photon Measurement Pipeline}
ECAL measurements, lepton conversions.
\subsection{Lepton Selection and Isolation}
Give iso and ID requirements, try to give justification for each
\subsection{Isolated Tracks}
Explain the selection and the rationale. The extra lepton veto could be good to mention here too. 
\subsection{MET Reconstruction} \label{sec:MET_reco}
  Make sure to add bits about sources of MET and the Type 1 correction
\subsection{Event Discriminators}
  We apply certain filters that kill events, these are called MET filters but include things like beam halo as well.

\section{Monte Carlo}