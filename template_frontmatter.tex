%
%
% UCSD Doctoral Dissertation Template
% -----------------------------------
% http://ucsd-thesis.googlecode.com
%
%


%% REQUIRED FIELDS -- Replace with the values appropriate to you

% No symbols, formulas, superscripts, or Greek letters are allowed
% in your title.

\title{A Supersymmetry Inspired Search for Dark Particles Coupled to the Z Boson with Jets, Missing Transverse Momentum, and Two Opposite-Charge Same-Flavor Leptons in Proton-Proton Collisions with 13 TeV Center of Mass Energy Using The Compact Muon Solenoid at The Large Hadron Collider}

\author{Bobak Hashemi}
\degreeyear{\the\year}

% Master's Degree theses will NOT be formatted properly with this file.
\degreetitle{Doctor of Philosophy}

\field{Physics}
\specialization{High Energy Experiment}  % If you have a specialization, add it here

\chair{Professor Frank W\"uerthwein}
% Uncomment the next line iff you have a Co-Chair
\cochair{Professor Avraham Yagil}
%
% Or, uncomment the next line iff you have two equal Co-Chairs.
%\cochairs{Professor Chair Masterish}{Professor Chair Masterish}

%  The rest of the committee members  must be alphabetized by last name.
\othermembers{
Professor John McGreevy\\
Professor David Meyer\\
Professor Daniel Arovas\\
}
\numberofmembers{5} % |chair| + |cochair| + |othermembers|


%% START THE FRONTMATTER
%
\begin{frontmatter}

%% TITLE PAGES
%
%  This command generates the title, copyright, and signature pages.
%
\makefrontmatter

%% DEDICATION
%
%  You have three choices here:
%    1. Use the ``dedication'' environment.
%       Put in the text you want, and everything will be formated for
%       you. You'll get a perfectly respectable dedication page.
%
%
%    2. Use the ``mydedication'' environment.  If you don't like the
%       formatting of option 1, use this environment and format things
%       however you wish.
%
%    3. If you don't want a dedication, it's not required.
%
%
\begin{dedication}
  To two, the loneliest number since the number one.
\end{dedication}


% \begin{mydedication} % You are responsible for formatting here.
%   \vspace{1in}
%   \begin{flushleft}
% 	To me.
%   \end{flushleft}
%
%   \vspace{2in}
%   \begin{center}
% 	And you.
%   \end{center}
%
%   \vspace{2in}
%   \begin{flushright}
% 	Which equals us.
%   \end{flushright}
% \end{mydedication}



%% EPIGRAPH
%
%  The same choices that applied to the dedication apply here.
%
\begin{epigraph} % The style file will position the text for you.
  \emph{A careful quotation\\
  conveys brilliance.}\\
  ---Smarty Pants
\end{epigraph}

% \begin{myepigraph} % You position the text yourself.
%   \vfil
%   \begin{center}
%     {\bf Think! It ain't illegal yet.}
%
% 	\emph{---George Clinton}
%   \end{center}
% \end{myepigraph}


%% SETUP THE TABLE OF CONTENTS
%
\tableofcontents
\listoffigures  % Comment if you don't have any figures
\listoftables   % Comment if you don't have any tables



%% ACKNOWLEDGEMENTS
%
%  While technically optional, you probably have someone to thank.
%  Also, a paragraph acknowledging all coauthors and publishers (if
%  you have any) is required in the acknowledgements page and as the
%  last paragraph of text at the end of each respective chapter. See
%  the OGS Formatting Manual for more information.
%
\begin{acknowledgements}
 
\end{acknowledgements}


%% VITA
%
%  A brief vita is required in a doctoral thesis. See the OGS
%  Formatting Manual for more information.
%
\begin{vitapage}
\begin{vita}
  \item[2012] B.~S. in Physics, Pennsylvania State University
  \item[2014] M.~S. in Physics, University of California, San Diego
  \item[2012-2015] Graduate Teaching Assistant, University of California, San Diego
  \item[2018] Ph.~D. in Physics, University of California, San Diego
\end{vita}
\begin{publications}
  \item M.W.E. Smith, et. al., \emph{The Astrophysical Multimessenger Observatory Network (AMON)}, Astroparticle Physics (2013) 2013: 45. https://doi.org/10.1016/j.astropartphys.2013.03.003.
  \item The CMS collaboration, Sirunyan, A.M., Tumasyan, A. et al., \emph{Search for new phenomena in final states with two opposite-charge, same-flavor leptons, jets, and missing transverse momentum in pp collisions at $\sqrt{s}=13$ TeV},  J. High Energ. Phys. (2018) 2018: 76. https://doi.org/10.1007/s13130-018-7845-2
  \item The CMS collaboration, Sirunyan, A.M., Tumasyan, A. et al., \emph{Combined search for electroweak production of charginos and neutralinos in proton-proton collisions at  $\sqrt{s}=13$ TeV}, J. High Energ. Phys. (2018) 2018: 160. https://doi.org/10.1007/JHEP03(2018)160
\end{publications}
\end{vitapage}


%% ABSTRACT
%
%  Doctoral dissertation abstracts should not exceed 350 words.
%   The abstract may continue to a second page if necessary.
%
\begin{abstract}
  This thesis presents the results of a search for new physics in proton-proton collisions at the Large Hadron Collider, running at $\sqrt{s} = 13$ TeV using data gathered by the Compact Muon Solenoid. The search uses a final state with two opposite-charge same-flavor light leptons (electrons or muons), at least 2 hadronic jets, and at least 100 GeV of missing transverse momentum. No statitsically significant deviation is found from the expected standard model background. The search results are interpreted in the context of several simplied model spectra (SMS) of supersymmetry, including a model of Gauge Mediated Supersymmetry-Breaking (GSMB) with gluino production and models with Electroweakino production. The limits on these models are advanced by several 100 GeV in mass parameters with respect to previous searches, and this work represents the current state of the art for their exclusion.
\end{abstract}


\end{frontmatter}
