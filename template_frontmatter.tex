%
%
% UCSD Doctoral Dissertation Template
% -----------------------------------
% http://ucsd-thesis.googlecode.com
%
%


%% REQUIRED FIELDS -- Replace with the values appropriate to you

% No symbols, formulas, superscripts, or Greek letters are allowed
% in your title.

%\title{A Supersymmetry Inspired Search for Dark Particles Coupled to the Z Boson with Jets, Missing Transverse Momentum, and Two Opposite-Charge Same-Flavor Leptons in Proton-Proton Collisions with 13 TeV Center of Mass Energy Using The Compact Muon Solenoid at The Large Hadron Collider}
\title{A Search for Dark Particles Produced in Association with the Z Boson and Jets in 13 TeV Proton-Proton Collisions}

\author{Bobak Hashemi}
\degreeyear{\the\year}

% Master's Degree theses will NOT be formatted properly with this file.
\degreetitle{Doctor of Philosophy}

\field{Physics}
%\specialization{High Energy Experiment}  % If you have a specialization, add it here

\chair{Professor Frank W\"uerthwein}
% Uncomment the next line iff you have a Co-Chair
\cochair{Professor Avraham Yagil}
%
% Or, uncomment the next line iff you have two equal Co-Chairs.
%\cochairs{Professor Chair Masterish}{Professor Chair Masterish}

%  The rest of the committee members  must be alphabetized by last name.
\othermembers{
Professor Daniel Arovas\\
Professor John McGreevy\\
Professor David Meyer\\
}
\numberofmembers{5} % |chair| + |cochair| + |othermembers|


%% START THE FRONTMATTER
%
\begin{frontmatter}

%% TITLE PAGES
%
%  This command generates the title, copyright, and signature pages.
%
\makefrontmatter

%% DEDICATION
%
%  You have three choices here:
%    1. Use the ``dedication'' environment.
%       Put in the text you want, and everything will be formated for
%       you. You'll get a perfectly respectable dedication page.
%
%
%    2. Use the ``mydedication'' environment.  If you don't like the
%       formatting of option 1, use this environment and format things
%       however you wish.
%
%    3. If you don't want a dedication, it's not required.
%
%
\begin{dedication}
  To my Mother and Father, who taught me to reach and keep reaching.
\end{dedication}


% \begin{mydedication} % You are responsible for formatting here.
%   \vspace{1in}
%   \begin{flushleft}
% 	To me.
%   \end{flushleft}
%
%   \vspace{2in}
%   \begin{center}
% 	And you.
%   \end{center}
%
%   \vspace{2in}
%   \begin{flushright}
% 	Which equals us.
%   \end{flushright}
% \end{mydedication}



%% EPIGRAPH
%
%  The same choices that applied to the dedication apply here.
%
\begin{epigraph} % The style file will position the text for you.
  \emph{The most incomprehensible thing about the universe is that it is comprehensible}\\
  ---Albert Einstein
\end{epigraph}

% \begin{myepigraph} % You position the text yourself.
%   \vfil
%   \begin{center}
%     {\bf Think! It ain't illegal yet.}
%
% 	\emph{---George Clinton}
%   \end{center}
% \end{myepigraph}


%% SETUP THE TABLE OF CONTENTS
%
\tableofcontents
\listoffigures  % Comment if you don't have any figures
\listoftables   % Comment if you don't have any tables



%% ACKNOWLEDGEMENTS
%
%  While technically optional, you probably have someone to thank.
%  Also, a paragraph acknowledging all coauthors and publishers (if
%  you have any) is required in the acknowledgements page and as the
%  last paragraph of text at the end of each respective chapter. See
%  the OGS Formatting Manual for more information.
%
\begin{acknowledgements}
Where to start with a project as large as a CMS analysis? The CMS collaboration is a group of thousands of scientists, engineers, technicians, students, and staff from over 200 institutes in over 40 countries. The collaboration built the CMS detector, and now operates and processes the data it delivers, allowing scientists like me to analyze the output of the proton-proton collisions. Without the thousands of people who have spent decades designing, troubleshooting, organizing, fund-raising, ect... the work in this dissertation would not be remotely possible. I thank my co-authors in the collaboration for their essential contributions to this work.

The CMS detector itself is located at CERN near Geneva Switzerland. Again here, we have thousands of people who have worked for decades ensuring that CMS has proton beams colliding in its belly at extremely high energy today. Again here it is impossible to overstate the absolute necessity of the LHC team and CERN staff to this work. 

Within the CMS collaboration, there are several individuals whose contributions to this work are much more direct. Pablo Martinez Ruiz del Arbol, Christian Schomakers, Leonora Vesterbacka, and Sergio Sánchez Cruz all contributed to this analysis, providing numerical factors, cross validations, discussions about methodology, and so on. Here I must also mention the tremendous leadership and mentorship provided to me by Dominick Olivito and Vince Welke, both of whom not only contributed code and man-hours, but also gave me direction, engaged me intellectually, and taught me just about everything I know about this search. These individuals, along with myself, constitute the Edge-Z team within the CMS SUSY group.

Next, I would like to thank the rest of the members of the CMS SUSY group for their important comments, feedback, and peer-review of the search, specifically Cristina Botta and Lesya Shchutska who were the conveners while this search was being executed. In addition to the SUSY group, we received important feedback from many members of CMS at large, and a review committee consisting of Marcello Maggi, Niki Saoulidou, Hongxuan Liu, and Christopher West scrutinized our search more thoroughly and provided crucial comments about the presentation of our methods.

I must also acknowledge SNT, an unofficial group of collaborators working within CMS including: P. Chang, M. Derdzinski, D. Gilbert, D. Klein, V. Krutelyov M. Masciovecchio, S. May, D. Olivito, F. Würthwein, A. Yagil, G. Zevi Della Porta, N. Amin, C. Campagnari, B. Marsh, S. Wang, L. Bauerdick, K. Burkett, O. Gutsche, S. Jindariani, M. Liu, H. Weber, F. Golf, I. Suarez. Almost all of the members of SNT have, to some degree, been important to the success of this project. Whether through discussion, contribution of code, or data processing, being part of the SNT team ensured I never could be stuck for too long on any issue.

One more time I would like to Dominick Olivito who has been both a cherished friend and irreplaceable mentor to me throughout my time at CMS. In addition to Dominick, my co-advisors Frank W\"uerthwein and Avraham Yagil were incredibly patient and good-natured PIs who trained me to be a particle physicist and never let me forget the importance of clear presentation skills, practice, and clear-thinking.

I would also like to thank Dylan Gilbert, Philip Chang, Mark Derdzinski, and Nick Amin for important discussions about physics, software, and work in HEP. Mark and Nick also contributed to this work by processing data from CMS into more accessible data formats for the SNT group at large. 

Finally, I would like to specifically thank the following senior members of SNT who have also been tremendous mentors to me: Giovani Zevi Della Porta, Philip Chang, Mia Liu, Hannsjorg Weber, Claudio Campagnari, Frank Golf, and Indara Suarez.

In chapter 2, I must also specifically thank my co-authors on CMS for contributing figure \ref{fig:csv_v2_performance} showing the B-tag CSV discriminator, figure \ref{fig:met_resolution} showing the MET resolution after the type-1 correction, and figure \ref{fig:met_filters_efficacy} showing the efficacy of the MET filters.

In chapter 3, I must also specifically thank my co-authors on CMS for contributing figure \ref{fig:t5zz_interpretation} (a), showing the previous best limits on the strong GMSB SUSY model.
\end{acknowledgements}


%% VITA
%
%  A brief vita is required in a doctoral thesis. See the OGS
%  Formatting Manual for more information.
%
\begin{vitapage}
\begin{vita}
  \item[2012] B.~S. in Physics, Pennsylvania State University
  \item[2014] M.~S. in Physics, University of California, San Diego
  \item[2012-2015] Graduate Teaching Assistant, University of California, San Diego
  \item[2018] Ph.~D. in Physics, University of California, San Diego
\end{vita}
\begin{publications}
  \item M.W.E. Smith, et. al., \emph{The Astrophysical Multimessenger Observatory Network (AMON)}, Astroparticle Physics (2013) 2013: 45. https://doi.org/10.1016/j.astropartphys.2013.03.003.
  \item The CMS collaboration, Sirunyan, A.M., Tumasyan, A. et al., \emph{Search for new phenomena in final states with two opposite-charge, same-flavor leptons, jets, and missing transverse momentum in pp collisions at $\sqrt{s}=13$ TeV},  J. High Energ. Phys. (2018) 2018: 76. https://doi.org/10.1007/s13130-018-7845-2
  \item The CMS collaboration, Sirunyan, A.M., Tumasyan, A. et al., \emph{Combined search for electroweak production of charginos and neutralinos in proton-proton collisions at  $\sqrt{s}=13$ TeV}, J. High Energ. Phys. (2018) 2018: 160. https://doi.org/10.1007/JHEP03(2018)160
\end{publications}
\end{vitapage}


%% ABSTRACT
%
%  Doctoral dissertation abstracts should not exceed 350 words.
%   The abstract may continue to a second page if necessary.
%
\begin{abstract}
  This thesis presents the results of a search for new physics in proton-proton collisions at the Large Hadron Collider, running with 13 TeV center of mass energy, using data gathered by the Compact Muon Solenoid (CMS). The search targets TeV mass-scale dark matter candidates and uses final states with two opposite-charge and same-flavor light leptons (electrons or muons) having dilepton mass consistent with the Z boson, at least 2 hadronic jets, and at least 100 GeV of transverse momentum imbalance. The thesis contains a historical introduction to particle physics, brief reviews of the standard model and supersymmetry, an in-depth discussion of the acquisition of data by the Large Hadron Collider and CMS detector, and a pedagogical overview of the analysis methods used. No statistically significant deviation is found from the expected standard model background. The search results are presented and interpreted in the context of several simplified models of supersymmetry, including a model of Gauge Mediated Supersymmetry-Breaking (GSMB) with gluino production and models with Electroweakino production. The excluded mass ranges for these models are advanced by approximately 50-100\% with respect to the best previous searches. This work represents the current state of the art for their exclusion.
\end{abstract}


\end{frontmatter}
