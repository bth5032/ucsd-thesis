\chapter{A search for new physics in events with a Z boson, missing transverse energy, and jets}

\section{Motivations}

This is some great motivation!

\section{Analysis Strategy}

\subsection{Background Considerations}

\subsubsection{Leptonic Final States}
The Z boson can decay to any fermion, in theory, one could perform this analysis in an all hadronic final state. It is true that the Z will decay to hadrons approximately 10 times as often as it will decay to light leptons. \todo{Add some figure for the Z decay rates, probably can site the PDG}.

However, in hadron colliders, the most common types of final states are those with hadronic jets, as can be seen in figure \ref{fig:lhc_decay_modes}; leptonic final states provide a much cleaner population in which to search for Z bosons. The decays of the Z produce two opposite sign, same flavor fermions. A Further bennefit of the leptonic channel is that flavor and charge identification is fairly easy for the light leptons but nearly impossible to identify for jets at the current state of the art (that is to say one can not say, for instance, one can not say with high confidence that a jet was produced by a positively charged charm quark). Finally, the energy and momentum measurements for the light leptons are much better than for jets because of the complexity involved in making good energy measurements for jets. \todo{Point to some discussion about hadron calorimetry, the basic idea here is just that jets have neutral particles and their contribution to the energy measurement needs to be implied by the reconstruction algorithm} 

Therefore, even though the decay rate to light leptons from Z bosons is lower than the production rate to hadronic final states, the better energy resolution and lower background rate makes the leptonic final state far more powerful.

When using the leptonic final states, there are essentially two other background sources of leptons we must consider, these are $\gamma$ and W decays. The W boson will decay into leptons only with a complimentary neutrino, this means that in order to select a pair of opposite charge and same flavor leptons, there must be at least two W bosons in the event. Because the decays of the W will be independent, there is only a 50\% chance that the two leptons in an event where two Ws decay leptonically will have the same flavor. As will be discussed later, this makes the background prediction for these types of events very easy as events with two same flavor leptons can be used to model essentially any kinematical distribution.

Depending on the source of the Ws, the kinematics of the leptons will differ. However, a general rule is that the total energy in the event will be peaked at some low value, near the threshold to produce the event, essentially the sum of the mass of all the prompt particles, and decay exponentially from there. Many kinematic distributions are highly correlated with the total energy in the event. In the case of the Z decay, we know the dilepton mass distribution will not be correlated

It turns out the most common source of W bosons is through the production of two top quarks, which decay to a bottom quark and a W Boson. To reject these events, we will use B-tagging, described in \todo{add reference to b-tagging section}. Further, the dilepton mass in these events will be essentially random \todo{Add a figure of WW and TTBar production MLL plot for ZMET. Is the distribution flat or falling?} but biased towards lower values. 

less likely, whereas for Z decay, the breit-wigner distribution predisposes the dilepton mass to be near 91 GeV. 

The overall dilepton mass distribuiton will be a falling distribution with a small bump for the Z Boson. 

\subsubsection{}

\section{Object And Event Selection}
  
  \subsection{Lepton Selection}

    \subsubsection{Trigger Requirements}

    \subsubsection{Lepton ID and Isolation}
    Transcribe ID and Iso requirements.

  \subsection{Search Regions}
    Electroweak search regions and strong search regions

\section{Background Estimation Methods}

\section{Results}

\section{Signal Interpretations}

\section{The Electroweak Combination}

The things we want to consider here are: 

  1. All the cuts and stuff, I want to list out what cuts we made and why we chose them. I am not sure where I will put all the data/MC agreement stuff. I guess that's mostly important for the MET profile as I don't really use MC in the rest of the
