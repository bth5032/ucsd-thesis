\chapter*{Glossary Of Terms} \label{ch:glossary}
\addcontentsline{toc}{chapter}{Glossary}

\newglossaryentry{HT}
{
        name=$H_T$,
        description={The scalar sum of the momentum in the transverse plane over all jets in an event. If there are two jets pointing at arbitrary angles with total energy $E_a=\sqrt{p_{a,x}^2 + p_{a,y}^2}$ and $E_b=\sqrt{p_{b,x}^2 + p_{b,y}^2}$ respectively, the $H_T$ for that event would be simply $E_a + E_B$}
}

\newglossaryentry{ISR_FSR}
{
        name=ISR/FSR,
        description={Initial and Final State Radiation}
}
\newglossaryentry{Light_lepton}
{
        name=Light Lepton,
        description={An electron or muon or their antiparticle partners.}
}
\newglossaryentry{OCSF}
{
        name=OCSF,
        description={Opposite charge, same flavor. Used in the context of events with two light leptons, meaning either $e^+e^-$ or $\mu^+ \mu^-$.}
}

\begin{itemize}

\item{$H_T$} The scalar sum of the momentum in the transverse plane over all jets in an event. If there are two jets pointing at arbitrary angles with total energy $E_a=\sqrt{p_{a,x}^2 + p_{a,y}^2}$ and $E_b=\sqrt{p_{b,x}^2 + p_{b,y}^2}$ respectively, the $H_T$ for that event would be simply $E_a + E_B$

\item{ISR/FSR} Initial and Final State Radiation

\item{Light lepton} An electron or muon or their antiparticle partners.

\item{OCSF} Opposite charge, same flavor. Used in the context of events with two light leptons, meaning either $e^+e^-$ or $\mu^+ \mu^-$.

\item{OCDF} Opposite charge, different flavor. Used in the context of events with two light leptons, meaning either $e^+\mu^-$ or $\mu^+ e^-$.

\item{prompt} A particle is called prompt if it was generated at the primary interaction vertex. A particle is called non-prompt if it was created in the decay of another particle further down the decay chain. 

\item{FromPV} From: https://twiki.cern.ch/twiki/bin/view/CMSPublic/WorkBookMiniAOD2015 The meaning of fromPV() results are unchanged, fromPV() returns a number between 3 and 0 to define how tight the association with the PV is: 
the tighest, 3 (PVUsedInFit), is if the track is used in the PV fit;
2 (PVTight) is if the track is not used in the fit of any of the other PVs, and is closest in z to the PV,
1 (PVLoose) is if the track is closest in z to a PV other then the PV.
0 (NoPV) is returned if the track is used in the fit of another PV.

\item{PVTight} See FromPV

\item{$M_T$} Transverse Mass, the mass of a 4 vector with the z-component set to 0.

\item{trigger efficiency}

\item{tracker muon}
\item{standalone muon}
\end{itemize}
