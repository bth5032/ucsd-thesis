\chapter{Conclusions}
In this thesis, we presented a search for dark particles which couple to the Z boson by searching in proton-proton collision final states with two opposite-charge and same-flavor light leptons, hadronic jets, and transverse momentum imbalance. No statistically significant deviation from the standard model predictions were found in the search regions. Interpretations of the search results were formulated with respect to simplified models of supersymmetry. The results presented represent the state-of-the-art for the exclusion of those simplified models.

For the model of GMSB, in figure \ref{fig:t5zz_diagram_interpretations}, gluino mass limits were extended by approximately 50\%, between 400-500 GeV, at similar neutralino masses. For the model with the WZ final state, in figure \ref{fig:tchiwz_diagram_interpretations}, electroweakino masses were extended 325 GeV from 275 GeV out to 600 GeV at low LSP masses, and LSP mass ranges excluded were extended out to 250 GeV. For the model with the ZZ final state, in figure \ref{fig:tchizz_diagram_interpretations}, the exclusion on the neutralino masses was extended from 375 GeV to 600 GeV. Finally, for the model with the HZ final state, in figure \ref{fig:tchihz_diagram_interpretations}, the exclusion on the neutralino mass was extended from 275 GeV to 500 GeV.

Though the standard model has enjoyed enormous success explaining and predicting the structure we see in fundamental interactions, observations of dark matter and energy from astrophysics, gravitation, and neutrino masses show that the standard model must not be the complete theory of nature. Supersymmetry has been distinguished as an extension to the standard model for several reasons, highly motivated by naturalness in models of TeV scale SUSY breaking, predicting new particles near the TeV scale. 

With no new particles found thus far and exclusion limits for simplified SUSY models pushing into the TeV scale, the theoretical consideration of naturalness and TeV scale SUSY breaking are becoming less and less tenable. However, there are still other indications, e.g. unification of forces and Higgs metastability\label{sec:theoretical_issues_with_sm}, that we expect to see some new particles in accessible mass ranges.

Other surprises can also be on the horizon. In the coming years we expect to see the birth of gravitational wave astronomy, multimessenger astronomy, the high luminosity LHC, and progress in the neutrino sector. All of these experimental avenues will test completely new regimes of physics and have promise to usher in a revolution of understanding about fundamental questions. The prevailing wisdom today among theoretical physicists is that the standard model is the low energy limit of a more complete theory. The community's ultimate goal is still to find a unified theory of everything that includes all known phenomenology including gravitation, the particles in the standard model, dark matter, and spacetime expansion. 