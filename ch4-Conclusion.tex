\chapter{Conclusions}

In this thesis, we presented a search for dark particles which couple to the Z boson by searching in proton-proton collision final states with two opposite charge/same flavor light leptons, hadronic jets, and transverse momentum imbalance. No statistically significant deviation from the standard model predictions were found in the search regions. Interpretations of the search results were formulated with respect to simplified models of supersymmetry. The results presented represent the state-of-the-art for the exclusion of these models.

Though the standard model has enjoyed enormous success and predicted much of the structure we see in fundamental interactions, observations from astrophysics show that the standard model must not be the complete theory of nature. With the no new particles found thus far at the TeV scale, the theoretical consideration of naturalness is becoming less and less tenable. However, it is still an interesting time to study fundamental physics. In the coming years we expect to see the birth of gravitational wave astronomy, multimessenger astronomy, the high luminosity LHC, and progress in the neutrino sector. 

The prevailing wisdom among theoretical physicists is that the standard model is the low energy limit of a grand unified theory, and the ultimate goal of foundational theoretical physics is to find the theory that describes gravitation, the standard model, dark matter, dark energy, and inflation. 